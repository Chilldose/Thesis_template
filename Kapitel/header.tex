\documentclass[
	pdftex,%              PDFTex verwenden
	a4paper,%             A4 Papier
	oneside,%             Einseitig
	bibtotoc,%    		Literaturverzeichnis einf�gen 
	%bibtotocnumbered%: nummeriert
	liststotoc,%		Verzeichnisse einbinden in toc
	idxtotoc,%            Index ins Verzeichnis einf�gen
	halfparskip,%        Europ�ischer Satz mit abstand zwischen Abs�tzen
%	chapterprefix,%       Kapitel anschreiben als Kapitel
	headsepline,%         Linie nach Kopfzeile
	%footsepline,%         Linie vor Fusszeile
	%pointlessnumbers,%     Nummern ohne abschlie�enden Punkt
	12pt%                 Gr�ssere Schrift, besser lesbar am bildschrim
]{scrbook}

 

%
% Paket f�r �bersetzungen ins Deutsche
%
\usepackage[french,ngerman,english]{babel}

%
% Pakete um Latin1 Zeichnens�tze verwenden zu k�nnen und die dazu
% passenden Schriften.
%
\usepackage[latin1]{inputenc}
\usepackage[T1]{fontenc}

%
% Paket f�r Quotes
%
\usepackage[babel,french=guillemets,german=swiss]{csquotes}

%
% Paket zum Erweitern der Tabelleneigenschaften
%
\usepackage{array}

%
% Paket f�r sch�nere Tabellen
%
\usepackage{booktabs}

%
% Paket um Grafiken einbetten zu k�nnen
%
\usepackage{graphicx}

%
% Paket f�r das richtige anzeigen von eingekreisten Zahlen
%
%\usepackage{tikz}
%\newcommand*\circled[1]{\tikz[baseline=(char.base)]{
%		\node[shape=circle,draw,inner sep=2pt] (char) {#1};}}
\newcommand*\circled[1]{\unitlength1ex\begin{picture}(2.5,2.5)%
	\put(0.75,0.75){\circle{3.2}}\put(0.7,0.7){\makebox(0,0){#1}}\end{picture}}

\usepackage{enumitem}

%
%Paket zum einstellen des Zeilenabstandes
%
\usepackage[singlespacing]{setspace}

%\usepackage[onehalfspacing]{setspace}

%\usepackage[doublespacing]{setspace}


%
% Damit links in der Literatur verwendet werden k�nnen
%
\usepackage{url}

%
% Spezielle Schrift im Koma-Script setzen.
%
\setkomafont{sectioning}{\normalfont\bfseries}
\setkomafont{captionlabel}{\normalfont\bfseries} 
\setkomafont{pagehead}{\normalfont\bfseries} % Kopfzeilenschrift
\setkomafont{descriptionlabel}{\normalfont\bfseries}

%
% Zeilenumbruch bei Bildbeschreibungen.
%
\setcapindent{1em}

%
% Kopf und Fu�zeilen
%
\usepackage{scrpage2}
\pagestyle{scrheadings}
% Inhalt bis Section rechts und Chapter links
\automark[section]{chapter}
% Mitte: leer
\chead{}

%
% mathematische symbole aus dem AMS Paket.
%
\usepackage{amsmath}
\usepackage{amssymb}

%
% Type 1 Fonts f�r bessere darstellung in PDF verwenden.
%
\usepackage{mathptmx}           % Times + passende Mathefonts
%\usepackage[scaled=.92]{helvet} % skalierte Helvetica als \sfdefault
%\usepackage{courier}            % Courier als \ttdefault

%
% Paket um Textteile drehen zu k�nnen
%
\usepackage{rotating}

%
% Paket f�r Farben im PDF
%
\usepackage{color}

%
% Paket f�r Links innerhalb des PDF Dokuments
%auch die der Inhaltslinks
%
\definecolor{LinkColor}{rgb}{0,0,0}
\usepackage[%
	pdftitle={Titel},% Titel der Diplomarbeit
	pdfauthor={Autor},% Autor(en)
	pdfcreator={LaTeX, LaTeX with hyperref and KOMA-Script},% Genutzte Programme
	pdfsubject={Betreff}, % Betreff
	pdfkeywords={Keywords}]{hyperref} % Keywords halt :-)
\hypersetup{colorlinks=true,% Definition der Links im PDF File
	linkcolor=LinkColor,%
	citecolor=LinkColor,%
	filecolor=LinkColor,%
	menucolor=LinkColor,%
	pagecolor=LinkColor,%
	urlcolor=LinkColor}

%
% Paket um LIstings sauber zu formatieren.
%
\usepackage[savemem]{listings}
\lstloadlanguages{TeX}

%
% Listing Definationen f�r PHP Code
%
\definecolor{lbcolor}{rgb}{0.85,0.85,0.85}
\lstset{language=[LaTeX]TeX,
	numbers=left,
	stepnumber=1,
	numbersep=5pt,
	numberstyle=\tiny,
	breaklines=true,
	breakautoindent=true,
	postbreak=\space,
	tabsize=2,
	basicstyle=\ttfamily\footnotesize,
	showspaces=false,
	showstringspaces=false,
	extendedchars=true,
	backgroundcolor=\color{lbcolor}}
%
% ---------------------------------------------------------------------------
%

%
% Neue Umgebungen
%
\newenvironment{ListChanges}%
	{\begin{list}{$\diamondsuit$}{}}%
	{\end{list}}

%
% aller Bilder werden im Unterverzeichnis figures gesucht:
%
\graphicspath{{bilder/}}

%
% Literaturverzeichnis-Stil
%
\usepackage[square, comma, numbers, sort&compress]{natbib}
\bibliographystyle{unsrt}

%
% Anf�hrungsstriche mithilfe von \textss{-anzufuehrendes-}
%
\newcommand{\textss}[1]{"`#1"'}

%
% Strukturiertiefe bis subsubsection{} m�glich
%
\setcounter{secnumdepth}{3}

%
% Dargestellte Strukturiertiefe im Inhaltsverzeichnis
%
\setcounter{tocdepth}{3}

%
% Zeilenabstand wird um den Faktor 1.5 ver�ndert
%
%\renewcommand{\baselinestretch}{1.5}

%
% Abk�rzungsverzeichnis
%
%\usepackage[intoc]{nomencl}
\usepackage{nomencl}
% Befehl umbenennen in abk
\let\abk\nomenclature
% Deutsche �berschrift
%\renewcommand{\nomname}{Abbreviations}
% Punkte zw. Abk�rzung und Erkl�rung
\setlength{\nomlabelwidth}{.20\hsize}
\renewcommand{\nomlabel}[1]{#1 \dotfill}
% Zeilenabst�nde verkleinern
\setlength{\nomitemsep}{-\parsep}
\makenomenclature


%
% Kapitel-�berschriften werden in eine Zeile geschrieben.
%
% normal: 
% Kapitel 1.
% Einleitung
%
% wenn hier auskommentiert:
% Kapitel 1: Einleitung
%
\usepackage{titlesec}
\titleformat{\chapter}[hang] 
{\normalfont\huge\bfseries}{\chaptertitlename\ \thechapter:}{1em}{} 

%Um acronyme verwenden zu k�nnen
\usepackage[printonlyused]{acronym}

%Bibliographie Alphabetisch oder Chronologisch ordnen
%\usepackage[backend=biber,style=alphabetic]{biblatex}

%Fue coole Farben
\usepackage[dvipsnames]{xcolor}

\definecolor{DarkGreen}{RGB}{0,128,0}
\definecolor{Tuerkis}{RGB}{0,255,255}
\definecolor{Red}{RGB}{255,0,0}

%damit man meherere bilder zusammen formatieren kann
\usepackage{subfigure}

%um gef�llte objekte zu erzeugen
\usepackage{tikz}

\newcommand{\filledcircle}[2][blue, fill=red]{\tikz[baseline=-1ex, line width=0.1ex]\draw[#1,radius=#2] (0,0) circle ;}
%usage \tikzcircle[framecolour, fill colour]{size}

\newcommand{\filleddiamond}[1][fill=black]{\tikz [x=2ex,y=2ex, line width=.1ex] \draw  [#1]  (0,.5) -- (.5,1) -- (1,.5) -- (.5,0) -- (0,.5) -- cycle;}%
%usage \filleddiamond[fill colour]

\newcommand{\filledstar}[1][fill=black]{\tikz [x=2ex,y=2ex, line width=.1ex] \draw  [#1]  (0.5,1) -- (0.65,0.6) -- (1,0.6) -- (0.7,0.4) -- (0.8,0) -- (0.5,0.3) -- (0.2,0) -- (0.3,0.4) -- (0,0.6) -- (0.35,0.6) -- cycle;}%


%\newcommand{\sqdiamond}[1][fill=black]{\tikz [x=1.2ex,y=1.85ex,line width=.1ex,line join=round, yshift=-0.285ex] \draw  [#1]  (0,.5) -- (.5,1) -- (1,.5) -- (.5,0) -- (0,.5) -- cycle;}%
%\newcommand{\sqdiamondDash}[1][fill=black]{%
%	\tikz [x=1.2ex,y=1.85ex,line width=.1ex,line join=round, yshift=-0.285ex] 
%	\draw  [#1]  
%	(0,.5) -- (.5,1) -- (1,.5) -- (.5,0) -- (0,.5) -- cycle
%	(0,1.1) --  (1,1.1);
%}%
%\newcommand{\MyDiamond}[1][fill=black]{\mathop{\raisebox{-0.275ex}{$\sqdiamond[#1]$}}}
%\newcommand{\MyDiamondDash}[1][fill=black]{\mathop{\raisebox{-0.275ex}{$\sqdiamondDash[#1]$}}}

%Um float objekte verwalten zu k�nnen
\usepackage{float}

%F�r subcaptions
%\usepackage{subcaption}

% Hack: Eigenen, rudiment�ren Kapitelstil definieren.

\usepackage{lmodern}

\setkomafont{chapter}{\huge}
\makeatletter
% Hack: Eigenen, rudiment�ren Kapitelstil definieren.
\newcommand*{\scr@dsc@def@style@xchapter@command}[1]{%
	\scr@dsc@def@style@chapter@command{#1}%
	\expandafter\def\csname @@make#1head\endcsname##1{%
		\let\orig@raggedchapter\raggedchapter
		\let\raggedchapter\raggedleft
		\let\orig@hangfrom\@hangfrom
		\let\@hangfrom\chapterhang
		\scr@@makechapterhead{#1}{##1}%
		\let\@hangfrom\orig@saved@hangfrom
		\let\raggedchapter\orig@raggedchapter
	}%
}
\makeatother
\DeclareSectionCommand[style=xchapter]{chapter}
\newbox\chapternumberbox
\newcommand{\chapterhang}[2]{%
	\savebox\chapternumberbox{\fontsize{100}{100}\normalfont\sffamily\bfseries
		\thechapter}%
	\parbox[t]{\dimexpr\linewidth-\wd\chapternumberbox-1em}{%
		\raggedleft
		\makebox[1em][l]{\normalfont\normalsize\slshape\chapapp}\\
		#2%
	}\quad
	\raisebox{-1.75\baselineskip}{\usebox\chapternumberbox}\par
} 
\usepackage{mwe}


%Damit auch das normale Layout bei kapitel funktioniert
\newcommand\normallayout{
	\titlespacing*{\chapter}{0pt}{50pt}{40pt}
	\titleformat{\chapter}[display]
	{\normalfont\huge\bfseries}{\chaptertitlename\ \thechapter}{20pt}{\Huge}
	\titleformat{\section}
	{\normalfont\Large\bfseries}{\thesection}{1em}{}
	\titleformat{\subsection}
	{\normalfont\large\bfseries}{\thesubsection}{1em}{}
}




%F�r die titelseite ein paket
\usepackage{epigraph}


\definecolor{pagecolor}{cmyk}{1,.60,0,.40}
\newcommand\pagedecoration{%
	\begin{tikzpicture}[remember picture,overlay,shorten >= -10pt]
	
	\coordinate (aux1) at ([yshift=-15pt]current page.north east);
	\coordinate (aux2) at ([yshift=-410pt]current page.north east);
	\coordinate (aux3) at ([xshift=-4.5cm]current page.north east);
	\coordinate (aux4) at ([yshift=-150pt]current page.north east);
	
	\begin{scope}[pagecolor!40,line width=12pt,rounded corners=12pt]
	\draw
	(aux1) -- coordinate (a)
	++(225:5) --
	++(-45:5.1) coordinate (b);
	\draw[shorten <= -10pt]
	(aux3) --
	(a) --
	(aux1);
	\draw[opacity=0.6,titlepagecolor,shorten <= -10pt]
	(b) --
	++(225:2.2) --
	++(-45:2.2);
	\end{scope}
	\draw[titlepagecolor,line width=8pt,rounded corners=8pt,shorten <= -10pt]
	(aux4) --
	++(225:0.8) --
	++(-45:0.8);
	\end{tikzpicture}%
}

